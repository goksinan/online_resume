% Copyright 2013 Christophe-Marie Duquesne <chmd@chmd.fr>
% Copyright 2014 Mark Szepieniec <http://github.com/mszep>
% 
% ConText style for making a resume with pandoc. Inspired by moderncv.
% 
% This CSS document is delivered to you under the CC BY-SA 3.0 License.
% https://creativecommons.org/licenses/by-sa/3.0/deed.en_US

\startmode[*mkii]
  \enableregime[utf-8]  
  \setupcolors[state=start]
\stopmode

\setupcolor[hex]
\definecolor[titlegrey][h=757575]
\definecolor[sectioncolor][h=397249]
\definecolor[rulecolor][h=9cb770]

% Enable hyperlinks
\setupinteraction[state=start, color=sectioncolor]

\setuppapersize [A4][A4]
\setuplayout    [width=middle, height=middle,
                 backspace=20mm, cutspace=0mm,
                 topspace=10mm, bottomspace=20mm,
                 header=0mm, footer=0mm]

%\setuppagenumbering[location={footer,center}]

\setupbodyfont[11pt, helvetica]

\setupwhitespace[medium]

\setupblackrules[width=31mm, color=rulecolor]

\setuphead[chapter]      [style=\tfd]
\setuphead[section]      [style=\tfd\bf, color=titlegrey, align=middle]
\setuphead[subsection]   [style=\tfb\bf, color=sectioncolor, align=right,
                          before={\leavevmode\blackrule\hspace}]
\setuphead[subsubsection][style=\bf]

\setuphead[chapter, section, subsection, subsubsection][number=no]

%\setupdescriptions[width=10mm]

\definedescription
  [description]
  [headstyle=bold, style=normal,
   location=hanging, width=18mm, distance=14mm, margin=0cm]

\setupitemize[autointro, packed]    % prevent orphan list intro
\setupitemize[indentnext=no]

\setupfloat[figure][default={here,nonumber}]
\setupfloat[table][default={here,nonumber}]

\setuptables[textwidth=max, HL=none]
\setupxtable[frame=off,option={stretch,width}]

\setupthinrules[width=15em] % width of horizontal rules

\setupdelimitedtext
  [blockquote]
  [before={\setupalign[middle]},
   indentnext=no,
  ]


\starttext

\section[title={Sinan Gok, PhD},reference={sinan-gok-phd}]

\thinrule

\startblockquote
\quotation{Work hard, be kind, and amazing things will happen.} Conan
O'Brien
\stopblockquote

\thinrule

\subsection[title={Key Skills},reference={key-skills}]

\startitemize[packed]
\item
  Over 6 years of experience developing medical instrumentation tools
  and techniques to collect real-time electrophysiological data
  (\useURL[url1][https://web.njit.edu/~sahin/][][NPL Lab]\from[url1]).
\item
  In-depth knowledge of brain-computer interfaces, neural control of
  movement, and mechanisms of neuromodulation.
\item
  Experience in biomedical signal processing, machine learning and
  predictive modelling, and exposure to statistical analysis techniques.
\item
  Ability to gather and interpret information, learn quickly, analyze
  data, manage a project, communicate with peers, present ideas, and
  think in an innovative and creative manner.
\item
  MATLAB, LabVIEW, Python, R, PTC Pro Engineer, Git/GitHub, Jekyll.
\stopitemize

\thinrule

\subsection[title={Education},reference={education}]

\startdescription{2012-2018}
  {\bf PhD, Biomedical Engineering}; New Jersey Institute of Technology
  (NJIT) & Rutgers University (Newark, NJ) - Joint PhD Program

  {\em Thesis title: Prediction of Forelimb Muscle Activities and
  Movement Phases Using the Corticospinal Signals in the Rat}
\stopdescription

\startdescription{2009-2011}
  {\bf MS, Electrical Engineering}; Lehigh University (Bethlehem, PA)
\stopdescription

\thinrule

\subsection[title={Experience},reference={experience}]

\startdescription{2018-2019}
  {\em Postdoctoral Research Associate,} NJIT

  \startitemize[packed]
  \item
    {\bf Project Management:} Planning and coordination of research
    activities including distribution of tasks, manufacturing devices,
    conducting experiments, collecting, analyzing and reporting data.
  \item
    {\bf Data Collection:} Utilizing NIdaq devices (National Instruments
    Inc.) to acquire electrophysiological signals in real time.
    Collecting video images (Allied Vision Inc.) for offline movement
    analysis.
  \item
    {\bf Data Analysis:} Offline analysis of experimental data to
    evaluate the effects of electrical stimulation on brain signals.
  \item
    {\bf Mentoring:} Supervising laboratory personnel.
  \stopitemize
\stopdescription

\startdescription{2012-2016}
  {\em Research Engineer,} NJIT/Rutgers

  \startitemize[packed]
  \item
    {\bf System Design:} Designed and built a data recording system for
    real time biological and video signal acquisition. Used PTC Pro
    Engineer to design and print 3D parts as needed. The system
    monitored a force sensor to automatically detect a behavioral task
    and store data using a FIFO buffer.
  \item
    {\bf Supervised Learning (Classification):} Classified forelimb
    extension and flexion movements using spinal cord signals recorded
    in rats. LDA, KNN, SVM, Naïve Bayes, and Random Forest algorithms
    were tested and compared for their classification performance.
  \item
    {\bf Supervised Learning (Regression):} Forelimb electromyography
    (EMG) signals were predicted from the spinal cord signals in rats.
    Multiple Linear Regression algorithm was used.
  \item
    {\bf Image Processing:} Video images of test subjects (rats)
    analyzed for movement identification. Custom written MATLAB scripts
    were used to read and analyze video recordings frame by frame.
  \item
    {\bf Academic:} Published research papers and presented academic
    work in national and international conferences. Tutored high school,
    undergraduate, and graduate students in summer research and school
    projects.
  \stopitemize
\stopdescription

\startdescription{Spring 2015}
  {\em Adjunct Instructor,} The College of New Jersey

  \startitemize[packed]
  \item
    {\bf Teaching:} Instructor for 400-level Bioinstrumentation
    Laboratory. Explained the principles of medical instrumentation such
    as, sampling, analog and digital signal processing, filtering and
    noise cancelling.
  \item
    {\bf Laboratory:} Taught how to use data acquisition systems and
    software tools to record and process biophysiological signals, such
    as ECG, EOG, and EMG.
  \stopitemize
\stopdescription

\thinrule

\subsection[title={Achievements},reference={achievements}]

\startitemize[packed]
\item
  {\bf Spinal cord injury research techniques training grant:} Sponsored
  by the New Jersey Commission on Spinal Cord Research (NJCSCR). April
  2017, Trenton, NJ.
\item
  {\bf NIH-supported summer course selectee:} Organized by National
  Center for Adaptive Neurotechnologies (NCAN). July 2017, Albany, NY.
\item
  {\bf Student travel award:} For giving a talk at the 38th IEEE EMBS
  conference. Sponsored by the Graduate Student Association at NJIT.
  October 2016, Newark, NJ.
\item
  {\bf Student travel award:} Sponsored by the NIH with support from BCI
  society and its affiliates. International Brain Computer Interface
  Meeting, May 2016, Pacific Grove, CA.
\item
  {\bf Graduate assistantship:} Biomedical Engineering Dept. at NJIT,
  2014-2018, Newark, NJ.
\item
  {\bf Full scholarship:} For Master's education in the US offered by
  the Turkish Ministry of Higher Education, 2009, Turkey.
\stopitemize

\thinrule

\subsection[title={Journal Articles},reference={journal-articles}]

\startitemize[packed]
\item
  {\bf S. Gok} and M. Sahin, \quotation{Prediction of forelimb EMGs and
  movement phases from corticospinal signals in the rat during the
  reach-to-pull task,} International Journal of Neural Systems,
  Feb.~2019 (In press).
\item
  A.S. Asan, {\bf S. Gok}, and M. Sahin, \quotation{Electrical Fields
  Induced Inside the Rat Brain with Skin, Skull, and Dural Placements of
  the Current Injection Electrode,} PLoS ONE, Jan.~2019.
\item
  Y. Guo, {\bf S. Gok}, and M. Sahin, \quotation{Convolutional networks
  outperform linear decoders in predicting EMG from spinal cord
  signals,} Frontiers in Neuroscience, 2018.
\stopitemize

\thinrule

\subsection[title={Conference Papers},reference={conference-papers}]

\startitemize[packed]
\item
  S. Asan, {\bf S. Gok}, and M. Sahin, \quotation{Electric fields
  induced by transcutaneous and intracranial current injections in the
  rat brain,} 40th Annual International Conference of the IEEE
  Engineering in Medicine and Biology Society (EMBC), Honolulu, HI,
  2018.
\item
  E. Cetinkaya, {\bf S. Gok}, and M. Sahin, \quotation{Carbon-fiber
  electrodes for in vivo spinal cord recordings,} 40th Annual
  International Conference of the IEEE Engineering in Medicine and
  Biology Society (EMBC), Honolulu, HI, 2018.
\item
  {\bf S. Gok} and M. Sahin, \quotation{Rat forelimb movement components
  segregated by corticospinal tract activity,} in International
  IEEE/EMBS Conference on Neural Engineering, NER, 2017, pp.~312--315.
\item
  {\bf S. Gok} and M. Sahin, \quotation{Prediction of forelimb muscle
  EMGs from the corticospinal signals in rats,} in Proceedings of the
  Annual International Conference of the IEEE Engineering in Medicine
  and Biology Society, EMBS, 2016, vol.~2016-- October, pp.~2780--2783.
  (Oral presentation)
\stopitemize

\thinrule

\subsection[title={Presentations},reference={presentations}]

\startitemize[packed]
\item
  E. Cetinkaya, {\bf S. Gok}, and M. Sahin. \quotation{Carbon fiber
  electrodes for recording spinal cord activity in rats,} presented at
  the BMES 2017 Annual Meeting, 2017, Phoenix, AZ.
\item
  {\bf S. Gok} and M. Sahin. \quotation{Predicting forelimb muscle
  activity from corticospinal signals in rats,} presented at the 6th
  International Brain Computer Interface Meeting, 2016, Pacific Grove,
  CA.
\item
  {\bf S. Gok}, H. Charkhkar, J. Pancrazio, and M. Sahin. \quotation{In
  vivo impedance characterization of PEDOT:TFB coated and chronically
  implanted multi electrode arrays,} presented at the BMES Annual
  Meeting, 2015, Tampa, FL.
\item
  {\bf S. Gok} and M. Sahin. \quotation{A method of chronic neural
  recording from rat cortico-spinal tract using flexible multi-electrode
  arrays,} presented at the 41st Neural Interfaces Conference June 2014,
  Dallas, TX.
\stopitemize

\thinrule

\subsection[title={Professional
Development},reference={professional-development}]

\startitemize[packed]
\item
  {\bf Springboard Data Science Bootcamp:} First half of an intensive
  program introducing data science. Online, 2018.
\item
  {\bf Spinal Cord Injury Research Methods Workshop:} W.M. Keck Center
  for Collaborative Neuroscience at Rutgers University, October 2017,
  Piscataway, NJ.
\item
  {\bf A Science and Technology Career Symposium:} \quotation{What can
  you be with a PhD?} NYU School of Medicine, November 2017, New York,
  NY
\item
  {\bf Workshop on Peer Review Process:} \quotation{You, Too, Can Peer
  Review!} by Angela Welch, PhD -- Elsevier, March 2017, New York
  Academy of Sciences, New York, NY
\item
  {\bf Grant Writing Workshop:} Presented by: Rick McGee, PhD., Rutgers
  Robert Wood Johnson Medical School, February 2017, Piscataway, NJ
\item
  {\bf Workshop on Neural Signal Processing Methods:} Organized by
  Rutgers Brain Health Institute, April 2016, NJIT, Newark, NJ
\item
  {\bf The First Annual Rutgers Brain Health Institute Symposium:}
  St.~Peter's University, October 2015, Jersey City, NJ
\item
  {\bf Fifth Annual Current Advances in Spinal Cord Injury Research
  Symposium:} Reynolds Family Spine Laboratory, May 2015, Newark, NJ
\item
  {\bf Workshop on Quantifying Structure in Large Neural Datasets:}
  Columbia University, September 2014, New York, NY
\stopitemize

\thinrule

\startblockquote
Sinan Gok • Bloomfield, NJ, USA
\useURL[url2][mailto:email@example.com][][email@example.com]\from[url2]
• +00 (0)00 000 0000 • XX years old\crlf
address - Mytown, Mycountry
\stopblockquote

\stoptext
